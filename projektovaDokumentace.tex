\documentclass[a4paper, 12pt, titlepage]{article}
\usepackage[utf8]{inputenc}
\usepackage[czech]{babel}
\usepackage[T1]{fontenc}
\usepackage{indentfirst}
\usepackage{mdwlist}
\usepackage{hyperref}
\usepackage{graphicx}
\usepackage{multicol}
\title{Projekt do PB173 \\ Bezpečné videokonferenční architektura}
\author{Tučňáčí kolektiv}
\date{\today}   

\begin{document}
\maketitle
\newpage
\tableofcontents
\newpage

\section{Motivace:}

Jak uz bylo naznačeno, dnes je človek odposloucháván všude. Co si programátor 
sam nenapiše, to nemá zabezpečené. A pokud se jako programátoři mame domluvit na 
hacknuti cele NSA,
musíme se domlouvat na šifrovanem kanálu bez možnosti odposlechu ze strany NSA.
Za vedlejší cíle se dá považovat cvičeni implementace kryptografických metod v 
"praxi" (ve vlastním kódu C/C++) a rozvinuti schopnosti spolupráce na většim 
projektu v kolektivu.

\section{Použite technologie:}
C/C++ s knihovnou PolarSSL, v nasem připade vetšinově C++, ale jelikož je 
PolarSSL psána v C, tak sekce kolem šifrovani nemohou používat mechanismy C++ 
(např. použití pole znaků místo objektu typu String).
Dále planujeme použít knihovny pro praci se sockety/HTTPS, knihovnu pro vlákna a 
události. Pro zpracování obrazu a zvuku využijeme doporučené knihovny ze 
cvičení.

Konkrétneji ke kryptografii: pro zajištení důvěrnosti, integrity i soukromí 
využijeme šifrovacího schématu encrypt-then-mac v podobe otevřené implementace 
Galois/Counter Mode (GCM),
která je dostupná v PolarSSL. Jako hashovaci funkci budeme používat SHA-512, 
symetrickou šifrovaci funkcí bude AES v již zmíněném módu GCM a pro ustanovení 
komunikace použijeme asymetrické
kryptografie v podobě RSA.

\section{Náš přístup k řešení:}

V projektu vystupují tři komponenty, mezi kterými je ustanovena komunikace.

\paragraph{Certifikační autorita}
Certifikační autorita (CA) je komponentou, která zajišťuje ověřování a 
distribuci certifikátů veřejných klíčů
potřebných pro navázání asymetricky šifrované komunikace mezi klienty. 
Certifikační autorita přijímá certifikáty od nově příchozích klientů, a na 
žádost vydává veřejné certifikáty klientů. Pro komunikaci s CA je zřízen jeden 
pár veřejného a  soukromého klíče, kerý je používán CA. Veřejný klíč je pro 
klienta známý. Komunikace je ustavena klientem, který po CA požaduje ověření 
veřejného klíče jiného klienta. Na tento dotaz CA odpovídá dohledáním 
příslušného veřejného klíče ve své databázi a odesláním certifikátu tohoto 
klíče.

\paragraph{Server}
Server je komponentou zaštiťující ustanovení komunikace mezi klienty. Komunikace 
probíhá pomocí jedsnosměrně šifrované komunikace od klienta k serveru. Server 
přijímá od klienta požadavky na přihlášení do systému a odesílá ostatním 
připojeným klientům jeho dostupnost. Následně očekává požadavky na spojení. 
Nakonec po odpojení odešle všem klientům požadavek na smazání klienta.
Server přijme požadavek na spojení od klienta ke klientu. Tento požadavek 
propaguje
k cílovému klientovi společně s údaji pro navázání komunikace se zdrojovým 
klientem. V případě přijetí pozitivní odpovědi odešle komunikační údaje cílového 
klienta zdrojovému a označí oba klienty za nedostupné. Pokud některý z klientů 
ukončí komunikaci odešle tuto informaci serveru a ten označí oba klienty za 
dostupné.

\paragraph{Klient}
Komponenta klient zajiťuje rozhraní pro přímou komunikaci dvou klientů, navázání 
komunikace a získávání veřejných klíčůnvystupujících v komunikaci. Klient se 
přpojí k serveru a přijme seznam připojených klientů.

\subparagraph{Klient požaduje připojení ke klientovi:}
Klient požaduje připojení ke klientovi:
Iniciální klient (IK) si vyžádá od CA veřejný klíč cílového klienta (CK) 
a zašle serveru požadavek na spojení. Po kladné odpovědi serveru a přijetí 
komunikačních údajů o CK. Následně navazuje komunikaci s CK. Odešle požadavek 
spoelečně s náhodně vygenerovaným klíčem a inicializačním vektorem pro 
symetrickou kryptografii zašifrovaný a podepsaný pomocí asymetrické kryptografie 
k CK. Pokud CK odpoví kladně, přijme a zpracuje symetrický klíč a inicializační 
vektor pro komunikaci od CK a IK je schopen pomocí
veřejného klíče CK ověřit autenticitu. Nakonec navazuje a přijímá 
komunikaci k resp. od CK šifrovanou pomocí
symetrické kryptografie. 

\subparagraph{Klient přijímá požadavek na komunikaci od serveru:}
CK přijme od serveru požadavek a informace o IK. Od CA si vyžádá 
certifikát veřejného klíče IK a odpoví na požadavek. Následně přijímá požadavek 
přímo od IK, které ověří pomocí jeho veřejného certifikátu. Přijme klíč pro 
komunikaci od IK, vygeneruje a odešle klíč pro komunikaci mezi CK a IK. Nakonec 
přijímá a otevírá přímé spojení mezi IK a CK.

\section{Naše API:}

Viz hlavičkové soubory na GitHubu včetně dokumentace v repozitáři po názvem 
PB173\_tucnaci dostupný zde: \url{https://github.com/LukeMcNemee/PB173\_tucnaci}

\end{document}
